\section{Questions/Areas of Investigation}
  \par The project raises many essential questions, including how such an app would address the issue of user privacy, whether the accuracy of collecting data for potential close contacts of such a system is precise, and whether it might be limited by environmental factors,...
  \subsection{Privacy Concerns?}
    \par People from different countries have distinguishable opinions of privacy during the coronavirus pandemic. In China and South Korea, the authorities are publishing the movements of people before they were diagnosed with SARS-CoV-2 by tracking down their steps using GPS phone tracking \parencite{SingTraceTogether}. As a consequence, these approaches slow down the COVID – 19 outbreak since people are monitored and forced to quarantine themselves at home. Meanwhile, such a mandatory tool would raise serious concerns in many countries. People would argue that there are ethical factors with mass government surveillance that need to be considered. What if the populace cannot overcome concerns over human and civil rights? It should be clarified that how the app would manage personal details such as the user’s name, location, phone numbers, and by that way the governments still can easily identify the potentially infected people. Additionally, as establishing a national database may enable undesirable corporate, considerations for the security guarantee should be taken into account.
  \subsection{How accurate are current technologies?}
    \par A GPS tracking surveillance system might seem like a good method, but the accuracy of data collection might be not exact. It is not easy as just identifying people have had direct exposure to patients or someone who potentially has SARS-CoV-2 since the reliability and precision of GPS technology remain in question. In China, some citizens report that even though they haven’t been in close contact with anyone with the virus, they find themselves stuck with “red code” that is applied to determine people are not allowed to move around freely or even return home \parencite{Questions2}. Another considerable concern is that there might be people who do not use smartphones or resist to use the app that will make serious challenges to the success of such a system.
  \subsection{COVID-19 in the outer environment?}
    \par The approach of this project raises crucial questions about the virus’s behavior and environmental factors. One unclear feature is exactly how long SARS-Cov- 2 can live outside the host. Some researches on similar SARS-Cov-2-like viruses (such as Mers) suggested they can last on surfaces for as long as 9 days if these viruses are not disinfected. Some can live up to 28 days in low-temperature environments \parencite{Questions3}. These factors may lead to a situation: “You can sit within a few meters of someone and not be at risk. Meanwhile, it looks like you can come into contact with a train seat previously occupied by someone with the virus many hours earlier and be at risk,” said Assoc. Prof. Hannah Fry of University College London \parencite{Questions4}. Although the app actually cannot track all objects touched by confirmed cases and anyone having symptoms of the disease, the project should propose a suitable method to find the highest number of people with a high chance of being infected.
  \subsection{Conclusion}
    \par In sum, the project should have an insight into important questions such as: What levels of privacy are people expected to sacrifice through using our app? What potential solution might be applied to improve the accuracy of data collection and enhance the effectiveness of the app? How can we overcome limitations and challenges which are relative to environmental factors and the gaps in the knowledge about the virus behavior? If our group is able to tackle these problems, we will make the project progress properly.


