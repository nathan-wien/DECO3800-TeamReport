\section{Questions/Areas of Investigation}
  \par The project raises many essential questions, including how such an app would address the issue of user privacy, whether the accuracy of collecting data for potential close contacts of such a system is precise, and whether it might be limited by environmental factors.

  \subsection{Privacy Concerns?}
    \par People from different countries have distinguishable opinions of privacy during the coronavirus pandemic. In China and South Korea, the authorities are publishing the movements of people before they were diagnosed with SARS-CoV-2 by tracking down their steps using GPS phone tracking \parencite{SingTraceTogether}. As a consequence, these approaches slow down the COVID – 19 outbreak since people are monitored and forced to quarantine themselves at home. Meanwhile, such a mandatory tool would raise serious concerns in many countries. People would argue that there are ethical factors with mass government surveillance that need to be considered. What if the populace cannot overcome concerns over human and civil rights? It should be clarified that how the app would manage personal details such as the user’s name, location, phone numbers, and by that way the governments still can easily identify the potentially infected people. Additionally, as establishing a national database may enable undesirable corporate, considerations for the security guarantee should be taken into account. Alternatively, using Bluetooth and QR code cloud-based systems for close contact logging could be the solution since these techniques are being used in reality to preserve users’ privacy. The app can give users the choice to opt-in or opt-out certain data that they allow or do not want the app to track. In this way, our app still provides services to users who opt-out and ensure users’ privacy safety.

  \subsection{How accurate are current technologies?}
    \par A GPS tracking surveillance system might seem like a good method, but the accuracy of data collection might be not exact. It is not easy as just identifying people have had direct exposure to patients or someone who potentially has SARS-CoV-2 since the reliability and precision of GPS technology remain in question. In China, some citizens report that even though they haven’t been in close contact with anyone with the virus, they find themselves stuck with “red code” that is applied to determine people are not allowed to move around freely or even return home \parencite{Questions2}. Another considerable concern is that there might be people who do not use smartphones or resist to use the app that will make serious challenges to the success of such a system. When it comes to the Bluetooth approach, our group faced questions about the technical aspect: Is it possible to make a wide-area-broadcasting Bluetooth signal generator? In our model, all businesses and public areas are required to have a mobile device with the same contact tracing app installed. Additionally, the range of the Bluetooth transmitter must be long enough to connect those specific devices to amplify the signal and cover the whole area so that if any user enters the area, their phone could log the code being broadcast in the area. Last but not least, our QR code approach also copes with difficulties such as controlling the crowd in public spaces (parks, beaches,...).
    
  \subsection{COVID-19 in the outer environment?}
    \par The approach of this project raises crucial questions about the virus’s behavior and environmental factors. One unclear feature is exactly how long SARS-Cov- 2 can live outside the host. Some researches on similar SARS-Cov-2-like viruses (such as Mers) suggested they can last on surfaces for as long as 9 days if these viruses are not disinfected. Some can live up to 28 days in low-temperature environments \parencite{Questions3}. These factors may lead to a situation: “You can sit within a few meters of someone and not be at risk. Meanwhile, it looks like you can come into contact with a train seat previously occupied by someone with the virus many hours earlier and be at risk,” said Assoc. Prof. Hannah Fry of University College London \parencite{Questions4}. Although the app actually cannot track all objects touched by confirmed cases and anyone having symptoms of the disease, the project should propose a suitable method to find the highest number of people with a high chance of being infected. When  it  comes  to  the  final  stage, multiple aspects of problemsrelated to thevirus’s behaviorremain unresolved. For the impact of thesecritical issues on the success of the project, the build team needs to investigate to find out what could we do to prevent environmental factors influencing  the efficiency  of our contact tracing application.

  \subsection{What features/services should the app offer?}
    \par To attract people to use, the app should offer features/services that other contact tracing apps do not have. User support during lockdown might be a good answer. Our group should figure out what people need during the quarantine in terms of finance, physical health, and mental health. The project also raised many essential questions: Do users agree to give more data if the app does benefit them more? Do users agree to install the app if the Bluetooth and QR code model is explained to them? For example, the TraceTogether app in Singapore had been installed by only 20\% of the total population \parencite{SingTraceTogether}. Meanwhile, such contact tracing apps require at least 40\% of the population to use its to make a high performance \parencite{Questions5}. If our group can assure users about what the app is doing, the success rate of this project will be enhanced. The  final  version  of  the  prototype  has proposed a complete solutionof how the application would operate tooffer usersthe most essential functionalities. As acomprehensive answer for what features should be integrated into a contact tracing application is subjective, the build team could also investigate to learn from  the  strengths  and  weaknesses  of  similar applications  such  as COVIDSafe and TraceTogether.

  \subsection{Conclusion}
    \par In sum, the project hadan insight into important questions such as: What levels of privacy are people expected to sacrifice through using our app? What potential solution mightbe applied to improve the accuracy of data collection and enhance the effectiveness of  the  app?  How  can  we  overcome  limitations  and  challenges  which  are  relative  to environmental factors and the gaps in the knowledge about the virus behavior? Thoughour group successfully tackledmost  ofthe identified problems,  critical  issues  related  to the environmental  factors  remainbig  challengesthat  could preventour  contact tracing applicationfrom progressing properly. For that reason, the build teamneeds to carry out more  researches  into the  behavior  of  the SARS-Cov-2virusas  well  astoconduct  more surveys, interviews to see if people agree to install our app and provide more of their data if doing so greatly benefits them more in COVID-19 prevention.


