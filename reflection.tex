\section{Team Reflection}
  \par As soon as the team was formed, our team had a meeting to make a final decision on which project to work on to the end of the semester. At first, each member briefly introduced their idea in the first assignment, which was the background research report. After that, everyone gave their own opinions to choose the project that best suits the group. Our selection criteria were that the hardware should not be the main concern because the backgrounds of our team members were software engineering. For that reason, Mobile or Web application would be a perfect choice and there was a project by a team member working on a Mobile application of tracking locations of users in order to minimize the spread of the pandemic disease. Moreover, COVID-19 is the main concern these days, and proposing an application that can solve a growing crisis would be a great advantage for the team.
  \par At the early stage, the approach was to minimize the pandemic was using GPS, which seemed to be an effective solution since the location of users and the time stamp could be stored in the system. However, after discussions, there are some weaknesses of GPS to apply to the project that we might take into account. The first one is user privacy. The success of this project depends mostly on cooperation among users. When people know that they are required to turn on GPS and their location is tracked all the time, it is not comfortable for them to use the application. Supposing if a user was tested positive for the virus and they did not turn on their Location Services, there would be profound consequences. The second weakness is that this method may not guarantee accuracy. GPS cannot get exactly when people are in close contact but it can track the locations where he/she has been to. As a result, Bluetooth signals would be a better option to preserve user privacy at the highest level and ensure the accuracy of contact distance. However, GPS was still kept to reduce the spreading of the virus in public places. Our team then decided to eliminate the usage of GPS in the project and came up with the idea of the Bluetooth signal generator which could be installed in public places such as supermarkets, restaurants, etc. This generator would connect to Bluetooth signals from customers’ mobile devices in that area. The data then be encrypted and we could know where the user who has been positively tested for the virus went by. With this method, we can both track the interaction between users and the locations. However, we realized that the Bluetooth connection is a two-way connection which means that this kind of generator cannot work perfectly as we expected when there were many people access to a place at the same time. Finally, we agreed to combine Bluetooth and QR code in the system to achieve the highest efficiency. With QR code, when entered a place, people had to in the queue with social distancing applied, each person needed to scan the QR Code before accessing that place. Currently, our team is still perfecting this model because there are still some minor issues.
  \par There is no doubt that COVID-19 has affected a lot in the way our team operates. It can be said that in order for a team to perform at its best, there is a need for direct interaction between members. Fortunately, five out of six members in our team live in the same student accommodation which is our advantage over other teams. We can easily meet each other and discuss if there is a need to revise the project. But it is not for that reason that we do not have online meetings every week to update the status of the assignment. We have a weekly meeting on Zoom in every studio class to discuss our ideas. Even in weeks without studio classes, we still try to hold a meeting by ourselves so that we can accelerate the progress of our project. Our team also created a shared folder on Google Drive and a private channel on Slack. The folder is where we upload all the important files namely reports, clips of the presentation, notes, etc. The Slack channel is our main means of communication. All the crucial announcements will be posted on this channel so that the team members can quickly track the progress of the project. Besides, we use the channel to share recent articles or reports that are relevant to our topic to this channel. The detailed information about COVID-19 is still vague so we must update the situation every day.
  \par At the beginning of each meeting, we will make time for each person to present shortly what they have researched in that week. There is a whiteboard on Zoom where we can take short notes during the meetings. After everyone in the team is done, we move on to solving problems that are existing in our model. At the same time, we also revise the comments from our tutors and classmates on our project. Their comments are very valuable because it gives us a more objective view on this issue. Moreover, it helped us to find out more about the users’ needs and expectations. Before the meeting is over, we divide the work based on the requirements of the next assessment. A deadline for an individual is set so that our work is not delayed. In the end, all the notes that have been taken during the meetings are posted to our channel on Slack so we can review later. Although our project is not perfect yet and there are still many issues that we had to tackle, the team has been and will be working hard on this virus tracking application. Healthcare, which is a major topic related to our application, is not our strength but we are still trying hard to develop our knowledge in this field and of course technically. The evidence is that each member is very responsible for the part that they are assigned. We also adhere to the rules set out to make sure the group works the most effectively.
