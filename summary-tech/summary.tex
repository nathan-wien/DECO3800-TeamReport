\hi{Technological Discussion}
  \par \ti{This part \tb{should} be in the \tb{Background Research} part}.
  \hii{Exisiting Technologies}
    \par There have been many different contact tracing solutions developed around the world. Here, we will mainly discuss solutions of which the technological implementation is widely published and open-source. These includes:
      \begin{itemize}
        \item Singapore's BlueTrace
        \item DP3T
        \item Apple \& Google Exposure Notification
      \end{itemize}

  \hii{Bluetooth vs GPS}
    \par Most contact tracing solutions rely on the use of mobile phones. Almost each and every mobile phone nowadays supports two different types of technology that can be utilize for contact tracing: Bluetooth and GPS.
    \par There are contact tracing applications that use GPS, such as Israel's \TheShield \cite{TheShield1} and South Korea's contact tracing app \cite{Korea1}. However, the GPS approach pose some significant issues.
    \begin{itemize}
      \item The first issue is accuracy. According to \cite{BlueTrace1}, in comparison with Bluetooth, GPS is far less accurate, especially in indoor environments. For instance, GPS cannot recognize people on different floors in the same building.
      \item The second issue is privacy concern. Having the application tracking their location all the time may discourage users from using the application.
    \end{itemize}
    \par On the other hand, Bluetooth is not only more accurate, but also allow users to better protect their location data. Therefore, it is the approach that we choose for our contact tracing application.

  \hii{Centralized vs Decentralized Contact Tracing}
    \par There are also two different approaches in how the collected contact tracing data manipulated:
      \begin{itemize}
        \item \tb{Centralized}: the contact tracing data on each phone is sent back to a centralized server (of a government or a health organization). On the server side, people who are exposed to COVID-19 can be identified.
        \item \tb{Decentralized}: the contact tracing data is encrypted on each phone before being sent to a centralized server. The centralized server only acts as a intermediary, collecting and broadcasting the data to other users. When the application receives the data from the server, it locally works out if its user is exposed to COVID-19 or not.
      \end{itemize}

    \par A notable example of a centralized contact tracing application is \BlueTrace. The government has full knowledge of the identity of the user \cite{BlueTrace1}, and whether they are exposed or not. However, this causes privacy concern among users, thus hampering the adoption of the application. Only 25\% of Singaporean installed the application \cite{UserAcceptance2}

    \par Some decentralized are being developed and refined, most notably DP3T - developed by a group of scientists from multiple pretigious universities around the world, and Exposure Notification - developed by Apple and Google. These system are significantly better at safeguarding user identity and exposure data, which can only be revealed if the user voluntarily do so. The details of how their systems work will be discussed in the later part.

  \hii{Protect User Identity with Random Ephemeral User-IDs}
    \par Random Ephemeral User-IDs are used in all three systems that we are discussing: \BlueTrace, \DPTTT, and \ExposureNotification. The purpose of this is to protect user's identity: if a user has multiple random IDs in a day, it is hard for attackers to track who is who. However, the execution is a bit different on each system.
    \par For \BlueTrace, according to their white paper \cite{BlueTrace1}, IDs are generated by the centralized server using \tb{symmetric-key encryption} and then are downloaded to each user's phone. The encryption key is kept in the server, so that the government can still work out the identity of each user. Apart from privacy concern, another issue is that this system requires regular Internet connection for the application to pull down new IDs from the server.
    \par On the other hand, for \DPTTT and \ExposureNotification, the application will generate the user-IDs locally using irreversible \tb{cryptographic hash functions} \cite{DP3T1} \cite{ExpoNoti2}. This means that, if one user-id of user $A$ is $id_{A}$, then \tb{only $A$ knows that $id_{A}$ belongs to him or her}. Further details will be discussed in later part.

  \hii{Human-in-the-loop vs Human-out-of-the-loop and Q\&A functionality}
    \par A contact tracing application does speed up the process of contact tracing, but it cannot completely replace human work. When the application detects that a user is exposed, it does not necessary means that the user is infected - this is called a ``false positive". This issue is discussed in \cite{BlueTrace1}. Further information may be required for the health authorities to decide what to do next. This is our rationale behind our Q\&A functionality: the health authorities can use the application as a communication platform with the user to obtain further information.

\hi{System Specification}
  \par \ti{This part \tb{may} go into the conceptual design part?}.

  \hii{Aims}
    \begin{itemize}
      \item Tracing human contact
      \item Tracing location
    \end{itemize}

    \par For COVID-19, tracing human contact is not enough - as COVID-19 can survive in the environment for a certain period of time (\tbi{needs reference here}). Therefore, we need a method to do location tracing without raising privacy concerns.

    \par Some criteria needs to be fulfilled:
    \begin{itemize}
      \item \tb{For tracing human contact}:
        \begin{itemize}
          \item The identity of the user has to be protected, and can only be reveal if the user deliberately publishes his or her data.
            \begin{itemize}
              \item Random ephemeral IDs can be used to conceal user's identity: The ever-changing nature of this approach will make it hard for attackers to trace out users. Each user will have a new user ID every 2 hours. (the duration is not fixed; requires further research/expert advice)
              \item Only the user can recognize his or her IDs: even if the server is exposed to security threats, or if the server is being monitored by any organization, the identity of the user will still not be leaked to anyone. This can be done by generating the IDs locally.
            \end{itemize}
          \item The chance of two people having the same ID at the same time must be extremely low: else we will have a ``false positive" case.
        \end{itemize}
      \item \tb{For tracing location}:
        \begin{itemize}
          \item To align with the user-business-government cooperation model that we discussed, we will use QR Code for location tracing purpose.
          \item All the QR Codes that the user scanned should be stored locally and encrypted.
          \item We should provide a way for businesses and the authorities to check if their customers have scanned the code or not. This can be done by asking the customer to rescan the code. The app will notify if the same QR code has been scanned in the last $x$ hours. After the period of $x$ hours, no further checking by the businesses or the authorities can be done. The location data will still be stored on the device and will only be published if the user agrees to do so. (the duration $x$ is set based on the type of business)
        \end{itemize}
      \item \tb{For both functionalities}:
        \begin{itemize}
          \item (some more discussions here)
        \end{itemize}
    \end{itemize}

  \hii{Flow}
    \hiii{Human-contact Tracing}
      \begin{itemize}
          \item The user IDs are randomly generated locally on each user's phone.
          \item When \tb{Alice} and \tb{Bob} come into close distance, the two mobile phones will exchange user IDs: Alice's phone will save a copy of Bob's ID and vice versa.
          \item Later, \tb{Bob} is diagnosed and confirmed by the health department to have COVID-19.
          \item \tb{Bob} is authorized by the health department to upload his ``list" of close contact to the server. It is more like a ``blackbox" than a list: no one can extract user IDs from it. When the app on \tb{Alice}'s phone has downloaded the ``blackbox" $B$ from the server, it will do comparison locally like this: For each ID $id_{i}$ of \tb{Alice}, the operation $B(id_i)$ will either output $true$ if the $id_i$ is in the blackbox, or $false$ if $id_i$ is not in the blackbox. If any ID of \tb{Alice} is in the blackbox, the app will notify her.
      \end{itemize}
      \par \ti{Similar solutions was proposed by \DPTTT and and \ExposureNotification. Here we only discuss the human-contact tracing function at a high-level without going into details}.
    \hiii{Location Tracing}
      \begin{itemize}
        \item \tb{When a user enters a business place}:
          \begin{itemize}
            \item When a user enters a business place, such as a supermarket, or a restaurant, he or she has to use the app to scan the QR Code at the entrance. This scan is valid for $h$ hours - where $h$ depends on the type of business.
            \item After $h$ hours, if the user wants to remain at the place, a rescan is required.
            \item During the period of $h$ hours, the business may ask the user to check if they have scanned the QR Code or not. They can do this by asking the user to rescan again. If there is a scan during the last $h$ hours, the app will notify.
            \item The police can do occassional drop-ins to check if the businesses follow the QR Code restriction. They can do this by asking the user to carry out the same process as mentioned above.
            \item The scanned QR Code stored on the device and is not sent to the centralized server.
          \end{itemize}
        \item \tb{When a user is confirmed to have COVID-19}
          \begin{itemize}
            \item The user will be authorized to upload the list of QR Codes to the server. The identity of the user is not included.
            \item The application on the phones of other users will periodically pull these lists down from the server and do comparison locally.
            \item When doing local comparison, if there is a matched location, the app will notify the user that he or she may have been exposed.
            \item The authorities, having full knowledge of the location list on the server site, will decide on the next action.
          \end{itemize}
      \end{itemize}
